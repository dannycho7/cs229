\item \subquestionpoints{8} \textbf{Cocktail Party Problem}

For this question you will implement the Bell and Sejnowski ICA algorithm, but
\textbf{assuming a Laplace source} (as derived in part-b), instead of the Logistic distribution
covered in class. The file \texttt{src/ica/mix.dat} contains the input data which consists of a matrix
with 5 columns, with each column corresponding to one of the mixed signals
$x_i$. The code for this question can be found in \texttt{src/ica/ica.py}.

First implement the \texttt{update\_W} and \texttt{unmix} functions in \texttt{src/ica/ica.py}. Then, you can run \texttt{ica.py} to split the mixed audio into its components. The mixed audio tracks are written to \texttt{mixed\_i.wav} in the output folder. The split audio tracks are written to \texttt{split\_i.wav} in the output folder. To make sure your code is correct, you should listen to the resulting unmixed sources. \\
\textbf{Finally, include the full unmixing matrix $W$ (5$\times$5) that you obtained in W.txt in your write-up.}\\
Notes:
\begin{enumerate}
  \item  Some overlap or noise in the sources may be present, but the different sources should be pretty clearly separated. If you implemention is correct, your output \texttt{split\_0.wav} should sound similar to the file \texttt{correct\_split\_0.wav} included with the source code.
  \item If your media player reports decoding error, try using the VLC media player.
  \item Mac users: iTunes may play the original mixed versions even if the program runs correctly. If your program doesn’t seem to be doing anything, consider renaming the saved sounds to slightly different filenames to double-check.
\end{enumerate}







